\documentclass{ximera}

%\usepackage{todonotes}

\newcommand{\todo}{}

\usepackage{tkz-euclide}
\tikzset{>=stealth} %% cool arrow head
\tikzset{shorten <>/.style={ shorten >=#1, shorten <=#1 } } %% allows shorter vectors

\usepackage{tkz-tab}  %% sign charts
\usetikzlibrary{decorations.pathreplacing} 

\usetikzlibrary{backgrounds} %% for boxes around graphs
\usetikzlibrary{shapes,positioning}  %% Clouds and stars
\usetikzlibrary{matrix} %% for matrix
\usepgfplotslibrary{polar} %% for polar plots
\usetkzobj{all}
\usepackage[makeroom]{cancel} %% for strike outs
%\usepackage{mathtools} %% for pretty underbrace % Breaks Ximera
\usepackage{multicol}

\usepackage{polynom}



\usepackage[many]{tcolorbox}  %% for titled boxes
\newtcolorbox{xbox}[1]{%
    tikznode boxed title,
    enhanced,
    arc=0mm,
    interior style={white},
    attach boxed title to top center= {yshift=-\tcboxedtitleheight/2},
    fonttitle=\bfseries,
    colbacktitle=white,coltitle=black,
    boxed title style={size=normal,colframe=white,boxrule=0pt},
    title={#1}}


\usepackage{array}
\setlength{\extrarowheight}{+.1cm}   
\newdimen\digitwidth
\settowidth\digitwidth{9}
\def\divrule#1#2{
\noalign{\moveright#1\digitwidth
\vbox{\hrule width#2\digitwidth}}}





\newcommand{\RR}{\mathbb R}
\newcommand{\R}{\mathbb R}
\newcommand{\N}{\mathbb N}
\newcommand{\Z}{\mathbb Z}

%\renewcommand{\d}{\,d\!}
\renewcommand{\d}{\mathop{}\!d}
\newcommand{\dd}[2][]{\frac{\d #1}{\d #2}}
\newcommand{\pp}[2][]{\frac{\partial #1}{\partial #2}}
\renewcommand{\l}{\ell}
\newcommand{\ddx}{\frac{d}{\d x}}
\newcommand{\ddt}{\frac{d}{\d t}}

\newcommand{\zeroOverZero}{\ensuremath{\boldsymbol{\tfrac{0}{0}}}}
\newcommand{\inftyOverInfty}{\ensuremath{\boldsymbol{\tfrac{\infty}{\infty}}}}
\newcommand{\zeroOverInfty}{\ensuremath{\boldsymbol{\tfrac{0}{\infty}}}}
\newcommand{\zeroTimesInfty}{\ensuremath{\small\boldsymbol{0\cdot \infty}}}
\newcommand{\inftyMinusInfty}{\ensuremath{\small\boldsymbol{\infty - \infty}}}
\newcommand{\oneToInfty}{\ensuremath{\boldsymbol{1^\infty}}}
\newcommand{\zeroToZero}{\ensuremath{\boldsymbol{0^0}}}
\newcommand{\inftyToZero}{\ensuremath{\boldsymbol{\infty^0}}}



\newcommand{\numOverZero}{\ensuremath{\boldsymbol{\tfrac{\#}{0}}}}
\newcommand{\dfn}{\textbf}
%\newcommand{\unit}{\,\mathrm}
\newcommand{\unit}{\mathop{}\!\mathrm}
\newcommand{\eval}[1]{\bigg[ #1 \bigg]}
\newcommand{\seq}[1]{\left( #1 \right)}
\renewcommand{\epsilon}{\varepsilon}
\renewcommand{\iff}{\Leftrightarrow}

\DeclareMathOperator{\arccot}{arccot}
\DeclareMathOperator{\arcsec}{arcsec}
\DeclareMathOperator{\arccsc}{arccsc}
\DeclareMathOperator{\si}{Si}
\DeclareMathOperator{\proj}{proj}
\DeclareMathOperator{\scal}{scal}


\newcommand{\tightoverset}[2]{% for arrow vec
  \mathop{#2}\limits^{\vbox to -.5ex{\kern-0.75ex\hbox{$#1$}\vss}}}
\newcommand{\arrowvec}[1]{\tightoverset{\scriptstyle\rightharpoonup}{#1}}
\renewcommand{\vec}{\mathbf}
\newcommand{\veci}{\vec{i}}
\newcommand{\vecj}{\vec{j}}
\newcommand{\veck}{\vec{k}}
\newcommand{\vecl}{\boldsymbol{\l}}

\newcommand{\dotp}{\bullet}
\newcommand{\cross}{\boldsymbol\times}
\newcommand{\grad}{\boldsymbol\nabla}
\newcommand{\divergence}{\grad\dotp}
\newcommand{\curl}{\grad\cross}
%\DeclareMathOperator{\divergence}{divergence}
%\DeclareMathOperator{\curl}[1]{\grad\cross #1}


\colorlet{textColor}{black} 
\colorlet{background}{white}
\colorlet{penColor}{blue!50!black} % Color of a curve in a plot
\colorlet{penColor2}{red!50!black}% Color of a curve in a plot
\colorlet{penColor3}{red!50!blue} % Color of a curve in a plot
\colorlet{penColor4}{green!50!black} % Color of a curve in a plot
\colorlet{penColor5}{orange!80!black} % Color of a curve in a plot
\colorlet{fill1}{penColor!20} % Color of fill in a plot
\colorlet{fill2}{penColor2!20} % Color of fill in a plot
\colorlet{fillp}{fill1} % Color of positive area
\colorlet{filln}{penColor2!20} % Color of negative area
\colorlet{fill3}{penColor3!20} % Fill
\colorlet{fill4}{penColor4!20} % Fill
\colorlet{fill5}{penColor5!20} % Fill
\colorlet{gridColor}{gray!50} % Color of grid in a plot

\newcommand{\surfaceColor}{violet}
\newcommand{\surfaceColorTwo}{redyellow}
\newcommand{\sliceColor}{greenyellow}




\pgfmathdeclarefunction{gauss}{2}{% gives gaussian
  \pgfmathparse{1/(#2*sqrt(2*pi))*exp(-((x-#1)^2)/(2*#2^2))}%
}


%%%%%%%%%%%%%
%% Vectors
%%%%%%%%%%%%%

%% Simple horiz vectors
\renewcommand{\vector}[1]{\left\langle #1\right\rangle}


%% %% Complex Horiz Vectors with angle brackets
%% \makeatletter
%% \renewcommand{\vector}[2][ , ]{\left\langle%
%%   \def\nextitem{\def\nextitem{#1}}%
%%   \@for \el:=#2\do{\nextitem\el}\right\rangle%
%% }
%% \makeatother

%% %% Vertical Vectors
%% \def\vector#1{\begin{bmatrix}\vecListA#1,,\end{bmatrix}}
%% \def\vecListA#1,{\if,#1,\else #1\cr \expandafter \vecListA \fi}

%%%%%%%%%%%%%
%% End of vectors
%%%%%%%%%%%%%

%\newcommand{\fullwidth}{}
%\newcommand{\normalwidth}{}



%% makes a snazzy t-chart for evaluating functions
%\newenvironment{tchart}{\rowcolors{2}{}{background!90!textColor}\array}{\endarray}

%%This is to help with formatting on future title pages.
\newenvironment{sectionOutcomes}{}{} 



%% Flowchart stuff
%\tikzstyle{startstop} = [rectangle, rounded corners, minimum width=3cm, minimum height=1cm,text centered, draw=black]
%\tikzstyle{question} = [rectangle, minimum width=3cm, minimum height=1cm, text centered, draw=black]
%\tikzstyle{decision} = [trapezium, trapezium left angle=70, trapezium right angle=110, minimum width=3cm, minimum height=1cm, text centered, draw=black]
%\tikzstyle{question} = [rectangle, rounded corners, minimum width=3cm, minimum height=1cm,text centered, draw=black]
%\tikzstyle{process} = [rectangle, minimum width=3cm, minimum height=1cm, text centered, draw=black]
%\tikzstyle{decision} = [trapezium, trapezium left angle=70, trapezium right angle=110, minimum width=3cm, minimum height=1cm, text centered, draw=black]


\outcome{Review derivatives.}

\title[Dig-In:]{Review Derivatives}


\begin{document}
\begin{abstract}
  Review differentiation.
\end{abstract}
\maketitle


\section{What are derivatives?}
After working with limits, our main tool last semester was the derivative.  We started with a function $f$ and a number $a$ in the
interior of its domain.  The derivative of $f$ at $a$ ( denoted as either $f'(a)$ or $\eval{ \ddx f(x)}_{x=a}$ ) can be interpreted
in a couple different ways.  It represents the \emph{instantaneous rate of change of $f$ at $a$}.  Graphically it represents
the \emph{slope of the line tangent to the graph of $y=f(x)$ at the point $(a, f(a) )$}. 

When we first came up with the idea, we didn't know how to calculate $f'(a)$ explicitly, so we approximated instead.  We
approximated the \emph{instantaneous rate of change at $a$} by the \emph{average rate of change on $[a, a+h]$}.  Graphically
we approximated the \emph{slope of the tangent line at $(a,f(a))$} by the \emph{slope of the secant line through $(a, f(a))$ and $(a+h, f(a+h))$}.
The exact value arose when we took the limit as $h$ tended to $0$.

The definition of the derivative is then:
\begin{definition}
  The \dfn{derivative} of $f$ is 
  \[
  \ddx f(x) = \lim_{h\to 0} \frac{f(x+h) - f(x)}{h}.
  \]
\end{definition}

Let's go through an example of using this definition.
\begin{example}
	Find an equation of the line tangent to the graph of  $f(x) = x^2 - 3x + 1$ at the point $(2, -1)$.
	\begin{explanation}
		To find an equation for a line we need a point on the line and the slope of the line.  Since we know the point
		$(2, -1)$ is on the line, our equation should look like:
		\[ y + 1 = m_{tan} (x-2) \]
		where $m_{tan}$ is the slope.  
		
		We know the slope at $(2, -1)$ means we need to find $f'\left( \answer[given]{2} \right)$.
		\begin{align*}
			f'(2) &= \lim_{h \to 0} \frac{ f(2+h) - f(2) }{h}\\
				&= \lim_{h \to 0} \frac{\left((2+h)^2 - 3(2+h) + 1\right) - \left(  -1 \right)}{h}\\
				&= \lim_{h \to 0} \frac{\left( (4 + 4h + h^2)-(6+3h) + 1\right) +1}{h}\\
				&= \lim_{h \to 0} \frac{h^2+h}{h}\\
				&= \lim_{h \to 0} \frac{h(h+1)}{h}\\
				&= \lim_{h \to 0} \frac{\cancel{h}(h+1)}{\cancel{h}}\\
				&= \lim_{h \to 0} (h+1)\\
				&= \answer{1}
		\end{align*}

		An equation for the line is: \[ y + 1 = 1 (x-2). \]
	\end{explanation}
\end{example}

\section{Shortcut formulas}
We soon turned to finding `shortcut formulas' for finding derivatives more easily.   The most important of these are in the table below.
\begin{xbox}{Power Rule}
\[ \ddx\left( x^n \right) = n x^{n-1} \]
\end{xbox}

\begin{xbox}{Combining Formulas}
	\begin{align*}
		\ddx \left( f(x) + g(x) \right) 	&= f'(x) + g'(x)\\
		\ddx \left( f(x) g(x) \right) 		&= f'(x) g(x) + f(x) g'(x)\\
		\ddx\left( \dfrac{f(x)}{g(x)} \right) &= \dfrac{f'(x)g(x) - f(x) g'(x)}{(g(x))^2}\\
		\ddx \left( f\left(g(x) \right) \right) &= f'\left( g(x) \right) g'(x)
	\end{align*}
\end{xbox}

\begin{xbox}{Exponentials and Logarithms}
	\begin{align*}
		\ddx \left( e^x \right) 		&= e^x\\
		\ddx \left( \ln x \right) 		&= \dfrac{1}{x}\\
		\\
		\ddx \left( b^x \right) 		&= b^x \ln b\\
		\ddx \left( \log_b x\right) 	&= \dfrac{1}{x \ln b}
	\end{align*}
\end{xbox}

\begin{xbox}{Trigonometric Functions}
	\begin{align*}
		\ddx\left( \sin x\right) 		&= \cos x\\
		\ddx \left( \cos x \right) 	&= - \sin x\\
		\ddx\left( \tan x \right) 	&= \sec^2 x\\
		\ddx \left( \sec x \right) 	&=  \sec x \tan x\\
		\ddx \left( \csc x \right) 	&= - \csc x \cot x\\
		\ddx \left( \cot x \right) 	&= - \csc^2 x
	\end{align*}
\end{xbox}

\begin{example}
	If an object has displacement function given by $s(t) = 2t^2 - 4t + e^t$, find the velocity at time $t = 2$.
	\begin{explanation}
		Since velocity is the rate of change of displacement, this is asking us to find $s'(2)$.  We'll find
		$v(t) = s'(t)$ by differentiating term-by-term.
		\begin{align*}
			v(t) 	&= \ddt\left(2t^2-4t+e^t \right)\\
				&= \ddt\left(2t^2\right) - \ddt\left(4t\right) + \ddt \left( e^t \right)\\
				&= \answer{4t} - \answer{4} + \answer{e^t}
		\end{align*}
		This gives a formula for $v(t)$, but we are being asked for the velocity at a specific time.  We still need to
		plug in $t = 2$.
		\[ v(2) = 8 - 4 + e^2 = 4+ e^2. \]
	\end{explanation}
\end{example}

\begin{example}
	Find $\ddx \left( 2x \tan \left(\log_5 (x+2)\right) \right)$.
	\begin{explanation}
		At its outer-most level, this is a multiplication problem.  The $2x$ is being multiplied by a tangent function.  That means we'll start with
		the Product Rule.
		\begin{align*} 
		\ddx \left( 2x \tan\left( \log_5 (x+2) \right) \right) &= \ddx(2x) \tan\left( \log_5 (x+2) \right) \\
			&+ (2x) \ddx\left( \tan\left( \log_5 (x+2)\right) \right).
		\end{align*}
		The first term on we can handle, since $\displaystyle \ddx(2x) = \answer{2}$.  The term on the right asks us to differentiate a tangent function, but the stuff
		on the inside isn't just the variable itself.  That means it is a Chain Rule problem.
			\[ \ddx \left( \tan \left( \log_5 (x+2) \right) \right) = \sec^2\left( \log_5 (x+2) \right) \cdot \ddx \left( \log_5 (x+2) \right). \]
		We have to take the derivative of a logarithm, but $x$ is not alone on the inside.  There is ANOTHER Chain Rule.
			\[ \ddx \left( \log_5 (x+2) \right) = \dfrac{1}{(x+2) \ln( \answer[given]{5} )} \cdot \ddx(x+2) = \dfrac{1}{(x+2) \ln 5}. \]

		Putting this all together, our derivative is:
		\[ 2 \tan \left( \log_5 (x+2) \right) + \dfrac{2x \sec^2 \left( \log_5 (x+2) \right)}{(x+2) \ln 5}. \]
	\end{explanation}
\end{example}


\begin{example}
	Suppose $f$ is a function whose values are given in the following table.
	\begin{center}
		\begin{tabular}{c c c}
			\hline \hline
			x & f(x) & f'(x) \\ 
			\hline
			0 & 1 & -1 \\
			1 & -1 & 3\\
			2 & 6 & -7\\
			\hline
		\end{tabular}		
	\end{center}
	
	Find $\displaystyle \eval{ \ddx\left( \cos\left( (x+\pi) f( x ) \right) \right)}_{x=0}$.
	\begin{explanation}
		Notice that, at its outer-most level, this function is just the cosine of something.  That means this is a Chain Rule problem.  Let's start with that.
		\[\ddx\left( \cos\left( (x+\pi) f( x ) \right) \right) = -\sin\left( (x+\pi) f( x ) \right) \cdot \ddx\left( (x+\pi) f( x ) \right). \]
		
		The stuff that was on the inside of the cosine is a product of $(x + \pi)$ with $f(x)$, so next is Product Rule.
		\begin{align*} 
			\ddx \left( x + \pi\right) f(x) &= \ddx\left(x+ \pi\right) f(x) + (x + \pi) \ddx \left( f(x) \right)\\
				&= \left(\answer[given]{1}\right) f(x) + (x+\pi) f'(x).
		\end{align*}
		Altogether we have:
		\[\ddx\left( \cos\left( (x+\pi) f( x ) \right) \right) = -\sin\left( (x+\pi) f( x ) \right) \cdot \left(f(x) + (x+\pi) f'(x) \right). \]
		
		Plugging in $x=0$ gives:
		\begin{align*}
			\eval{\ddx\left( \cos\left( (x+\pi) f( x ) \right) \right)}_{x=0} &= -\sin\left( \pi f( 0 ) \right) \cdot \left(f(0) + \pi f'(0) \right) \\
				&= -\sin\left( \pi \right) \cdot \left( 1 - \pi \right)\\
				&= \answer{0}.
		\end{align*}
	\end{explanation}
\end{example}


\section{Applications}
We finished last semester with a few applications that showed some of the things the derivative was good for.  (There are MANY others!)

Recall that a \emph{critical point} of a function $f$ is a number $c$  with either $f'(c)=0$ or $f'(c)$ does not exist.
\begin{question}
	Find the critical points of the function $\displaystyle f(x) = x e^x$.
	
	\begin{prompt} \[x = \answer{-1}\]\end{prompt}
\end{question}

Critical points played a large role for us in graphing functions and in optimization problems. Try the following optimization problem.
\begin{question}
	Among all rectangles with perimeter $30$, find the maximum area possible.
	
	
	\begin{prompt} \[ \textrm{The maximum area is } \answer{\frac{225}{4}}. \] \end{prompt}
\end{question}

\end{document}
