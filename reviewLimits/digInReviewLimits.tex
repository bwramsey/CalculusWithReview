\documentclass{ximera}

%\usepackage{todonotes}

\newcommand{\todo}{}

\usepackage{tkz-euclide}
\tikzset{>=stealth} %% cool arrow head
\tikzset{shorten <>/.style={ shorten >=#1, shorten <=#1 } } %% allows shorter vectors

\usepackage{tkz-tab}  %% sign charts
\usetikzlibrary{decorations.pathreplacing} 

\usetikzlibrary{backgrounds} %% for boxes around graphs
\usetikzlibrary{shapes,positioning}  %% Clouds and stars
\usetikzlibrary{matrix} %% for matrix
\usepgfplotslibrary{polar} %% for polar plots
\usetkzobj{all}
\usepackage[makeroom]{cancel} %% for strike outs
%\usepackage{mathtools} %% for pretty underbrace % Breaks Ximera
\usepackage{multicol}

\usepackage{polynom}



\usepackage[many]{tcolorbox}  %% for titled boxes
\newtcolorbox{xbox}[1]{%
    tikznode boxed title,
    enhanced,
    arc=0mm,
    interior style={white},
    attach boxed title to top center= {yshift=-\tcboxedtitleheight/2},
    fonttitle=\bfseries,
    colbacktitle=white,coltitle=black,
    boxed title style={size=normal,colframe=white,boxrule=0pt},
    title={#1}}


\usepackage{array}
\setlength{\extrarowheight}{+.1cm}   
\newdimen\digitwidth
\settowidth\digitwidth{9}
\def\divrule#1#2{
\noalign{\moveright#1\digitwidth
\vbox{\hrule width#2\digitwidth}}}





\newcommand{\RR}{\mathbb R}
\newcommand{\R}{\mathbb R}
\newcommand{\N}{\mathbb N}
\newcommand{\Z}{\mathbb Z}

%\renewcommand{\d}{\,d\!}
\renewcommand{\d}{\mathop{}\!d}
\newcommand{\dd}[2][]{\frac{\d #1}{\d #2}}
\newcommand{\pp}[2][]{\frac{\partial #1}{\partial #2}}
\renewcommand{\l}{\ell}
\newcommand{\ddx}{\frac{d}{\d x}}
\newcommand{\ddt}{\frac{d}{\d t}}

\newcommand{\zeroOverZero}{\ensuremath{\boldsymbol{\tfrac{0}{0}}}}
\newcommand{\inftyOverInfty}{\ensuremath{\boldsymbol{\tfrac{\infty}{\infty}}}}
\newcommand{\zeroOverInfty}{\ensuremath{\boldsymbol{\tfrac{0}{\infty}}}}
\newcommand{\zeroTimesInfty}{\ensuremath{\small\boldsymbol{0\cdot \infty}}}
\newcommand{\inftyMinusInfty}{\ensuremath{\small\boldsymbol{\infty - \infty}}}
\newcommand{\oneToInfty}{\ensuremath{\boldsymbol{1^\infty}}}
\newcommand{\zeroToZero}{\ensuremath{\boldsymbol{0^0}}}
\newcommand{\inftyToZero}{\ensuremath{\boldsymbol{\infty^0}}}



\newcommand{\numOverZero}{\ensuremath{\boldsymbol{\tfrac{\#}{0}}}}
\newcommand{\dfn}{\textbf}
%\newcommand{\unit}{\,\mathrm}
\newcommand{\unit}{\mathop{}\!\mathrm}
\newcommand{\eval}[1]{\bigg[ #1 \bigg]}
\newcommand{\seq}[1]{\left( #1 \right)}
\renewcommand{\epsilon}{\varepsilon}
\renewcommand{\iff}{\Leftrightarrow}

\DeclareMathOperator{\arccot}{arccot}
\DeclareMathOperator{\arcsec}{arcsec}
\DeclareMathOperator{\arccsc}{arccsc}
\DeclareMathOperator{\si}{Si}
\DeclareMathOperator{\proj}{proj}
\DeclareMathOperator{\scal}{scal}


\newcommand{\tightoverset}[2]{% for arrow vec
  \mathop{#2}\limits^{\vbox to -.5ex{\kern-0.75ex\hbox{$#1$}\vss}}}
\newcommand{\arrowvec}[1]{\tightoverset{\scriptstyle\rightharpoonup}{#1}}
\renewcommand{\vec}{\mathbf}
\newcommand{\veci}{\vec{i}}
\newcommand{\vecj}{\vec{j}}
\newcommand{\veck}{\vec{k}}
\newcommand{\vecl}{\boldsymbol{\l}}

\newcommand{\dotp}{\bullet}
\newcommand{\cross}{\boldsymbol\times}
\newcommand{\grad}{\boldsymbol\nabla}
\newcommand{\divergence}{\grad\dotp}
\newcommand{\curl}{\grad\cross}
%\DeclareMathOperator{\divergence}{divergence}
%\DeclareMathOperator{\curl}[1]{\grad\cross #1}


\colorlet{textColor}{black} 
\colorlet{background}{white}
\colorlet{penColor}{blue!50!black} % Color of a curve in a plot
\colorlet{penColor2}{red!50!black}% Color of a curve in a plot
\colorlet{penColor3}{red!50!blue} % Color of a curve in a plot
\colorlet{penColor4}{green!50!black} % Color of a curve in a plot
\colorlet{penColor5}{orange!80!black} % Color of a curve in a plot
\colorlet{fill1}{penColor!20} % Color of fill in a plot
\colorlet{fill2}{penColor2!20} % Color of fill in a plot
\colorlet{fillp}{fill1} % Color of positive area
\colorlet{filln}{penColor2!20} % Color of negative area
\colorlet{fill3}{penColor3!20} % Fill
\colorlet{fill4}{penColor4!20} % Fill
\colorlet{fill5}{penColor5!20} % Fill
\colorlet{gridColor}{gray!50} % Color of grid in a plot

\newcommand{\surfaceColor}{violet}
\newcommand{\surfaceColorTwo}{redyellow}
\newcommand{\sliceColor}{greenyellow}




\pgfmathdeclarefunction{gauss}{2}{% gives gaussian
  \pgfmathparse{1/(#2*sqrt(2*pi))*exp(-((x-#1)^2)/(2*#2^2))}%
}


%%%%%%%%%%%%%
%% Vectors
%%%%%%%%%%%%%

%% Simple horiz vectors
\renewcommand{\vector}[1]{\left\langle #1\right\rangle}


%% %% Complex Horiz Vectors with angle brackets
%% \makeatletter
%% \renewcommand{\vector}[2][ , ]{\left\langle%
%%   \def\nextitem{\def\nextitem{#1}}%
%%   \@for \el:=#2\do{\nextitem\el}\right\rangle%
%% }
%% \makeatother

%% %% Vertical Vectors
%% \def\vector#1{\begin{bmatrix}\vecListA#1,,\end{bmatrix}}
%% \def\vecListA#1,{\if,#1,\else #1\cr \expandafter \vecListA \fi}

%%%%%%%%%%%%%
%% End of vectors
%%%%%%%%%%%%%

%\newcommand{\fullwidth}{}
%\newcommand{\normalwidth}{}



%% makes a snazzy t-chart for evaluating functions
%\newenvironment{tchart}{\rowcolors{2}{}{background!90!textColor}\array}{\endarray}

%%This is to help with formatting on future title pages.
\newenvironment{sectionOutcomes}{}{} 



%% Flowchart stuff
%\tikzstyle{startstop} = [rectangle, rounded corners, minimum width=3cm, minimum height=1cm,text centered, draw=black]
%\tikzstyle{question} = [rectangle, minimum width=3cm, minimum height=1cm, text centered, draw=black]
%\tikzstyle{decision} = [trapezium, trapezium left angle=70, trapezium right angle=110, minimum width=3cm, minimum height=1cm, text centered, draw=black]
%\tikzstyle{question} = [rectangle, rounded corners, minimum width=3cm, minimum height=1cm,text centered, draw=black]
%\tikzstyle{process} = [rectangle, minimum width=3cm, minimum height=1cm, text centered, draw=black]
%\tikzstyle{decision} = [trapezium, trapezium left angle=70, trapezium right angle=110, minimum width=3cm, minimum height=1cm, text centered, draw=black]


\outcome{Review limits.}

\title[Dig-In:]{Review Limits.}


\begin{document}
\begin{abstract}
  Review methods of evaluating limits.
\end{abstract}
\maketitle


\section{Remember limits?}

We started last semester with the idea of a `limit'.  Remember what that means.
\begin{definition}
  Intuitively,
  \begin{center}
    the \dfn{limit} of $f(x)$ as $x$ approaches $a$ is $L$,
  \end{center}
  written
  \[
  \lim_{x\to a} f(x) = L,
  \]
  if the value of $f(x)$ can be made as close as one wishes to $L$ for
  all $x$ sufficiently close, but not equal, to $a$.
\end{definition}
If we look at the graph of a function $f$:
\begin{image}
\begin{tikzpicture}
	\begin{axis}[
            domain=1.5:2.5,
            ymin=-.5,ymax=3.5,
            axis lines =middle, xlabel=$x$, ylabel=$y$,
            every axis y label/.style={at=(current axis.above origin),anchor=south},
            every axis x label/.style={at=(current axis.right of origin),anchor=west},
          ]
	  \addplot [very thick, penColor, domain=(1:2)] {1};
          \addplot [very thick, penColor, domain=(2:3)] {2};
          \addplot[color=penColor,fill=penColor,only marks,mark=*] coordinates{(2,2)};  %% closed hole  
          \addplot[color=penColor,fill=background,only marks,mark=*] coordinates{(2,1)};  %% open hole
          
          %\addplot[(-),color=penColor2,ultra thick] plot coordinates {(1.75,0) (2.25,0)};

          %\addplot[color=penColor2,fill=background,only marks,mark=*] coordinates{(2,0)};  %% open hole
          %\addplot[textColor,dashed] plot coordinates {(1.9,0) (1.9,1)};
        \end{axis}
\end{tikzpicture}
\end{image}
If $x$ is really close to $a = 2.5$ (say, within $0.001$), then the output value is $f(x) = 2$.  As long as $x$ is within
a very small tolerance of $2.5$, the output values are always $2$.  That is, $\displaystyle \lim_{x\to 2.5} f(x) = 2$.

What happens at $a = 2$?  If $x$ is really close to $2$, but larger than $2$, then the output values $f(x) = 2$.  If
$x$ is really close to $2$, but less than $2$, then the output values $f(x) = 1$.  This lead us to consider 
`one-sided limits'.  We have $\displaystyle \lim_{x\to 2^+} f(x) = 2$ and $\displaystyle \lim_{x\to 2^-} f(x) = 1$.  (Does it
matter that the circle at $(2,1)$ is not filled in?)
What can we say about $\displaystyle \lim_{x\to 2} f(x)$?
\begin{question}
	What is $\displaystyle \lim_{x\to 2} f(x)$?
	\begin{multipleChoice}
		\choice{$1$}
		\choice{$2$}
		\choice{$1.5$}
		\choice[correct]{DNE}
	\end{multipleChoice}
\end{question}


We discussed several different methods for calculating limit values last semester, based on what we called the `Limit Laws'. We'll recall them here.
\begin{theorem}[Limit Laws]
Suppose that $\lim_{x\to a}f(x)=L$, $\lim_{x\to a}g(x)=M$.
	\begin{description}
		\item[Sum/Difference Law] $\lim_{x\to a} (f(x) \pm g(x)) = \lim_{x\to a}f(x) \pm \lim_{x\to a}g(x)=L \pm M$.
		\item[Product Law]  $\lim_{x\to a} (f(x)g(x)) = \lim_{x\to a}f(x)\cdot\lim_{x\to a}g(x)=LM$.
		\item[Quotient Law]  $\lim_{x\to a} \frac{f(x)}{g(x)} = \frac{\lim_{x\to a}f(x)}{\lim_{x\to a}g(x)}=\frac{L}{M}$, if $M\ne0$.
	\end{description}
\end{theorem}
These Limit Laws allow us to use limit values that we know, in order to calculate limit values of more complicated functions.

For example, if we know $\displaystyle \lim_{x\to \pi} x = \pi$ and $\displaystyle \lim_{x\to \pi} \cos(x) = -1$, then we can find
something like $\displaystyle \lim_{x\to \pi} \left(3x^2 - 4x \cos(x)\right)$.
\begin{example}
Calculate $\displaystyle \lim_{x\to \pi} \left(3x^2 - 4x \cos(x)\right)$.
	\begin{explanation}
		\begin{align*}
			\lim_{x\to \pi} \left(3x^2 - 4x \cos(x)\right) &= \lim_{x\to \pi} 3x^2 - \lim_{x\to \pi} 4x \cos(x)\\
				&= 3 \left(\lim_{x\to \pi} x \right)^2 - 4 \left( \lim_{x\to \pi} x \right) \left( \lim_{x\to \pi} \cos(x) \right)) \\
				&= 3 \left( \answer{\pi} \right)^2 - 4 \left( \answer{\pi} \right) \left( \answer{-1} \right)\\
				&= 3\pi^2 + 4\pi.
		\end{align*}
	\end{explanation}
\end{example}


From the Limit Laws, we saw that polynomial functions had a very important property for us.  They were `continuous'.
\begin{definition}
  A function $f$ is \dfn{continuous at a point} $a$ if
  \[
  \lim_{x\to a}f(x) = f(a).
  \]
\end{definition}
That means to find the limit value we just have to `plug in $a$'.  Most of our favorite functions have this property.  Polynomials and exponential functions are continuous everywhere.
Rational functions and trigonometric functions are continuous on their domain.  Inverse functions of invertible continuous functions are continuous on their domain, so logarithms and
radical functions are continuous on their domains.  If we take a composition of continuous functions, the result is also a continuous function!  Continuity gives us an efficient way of calculating
limits!

\begin{example}
	Evaluate: $\displaystyle \lim_{x\to 0} \left(x^2 \sin(x) + 4x\cos(x) - 3^{6x} \right)$
	\begin{explanation}
		The functions given by the equations $4x$, $x^2$, $\sin(x)$, and $\cos(x)$ are basic continuous functions.  The function $3^{4x}$ is a composition
		of the continuous functions $3^x$ and $6x$, so it is continuous.  The function $x^2 \sin(x) + 4x\cos(x) - 3^{6x}$ is itself a continuous function.  That means
		\[ \lim_{x\to 0} \left( x^2 \sin(x) + 4x \cos(x) - 3^{6x} \right) = \answer{-1}. \]
	\end{explanation}
\end{example}

Try one on your own.
\begin{problem}
	Evaluate the limit.
	\[ \lim_{x \to 3} \left( 4(x-2)^2 - 3e^{2x-6}+ \tan\left( \frac{x^2-8}{4} \pi \right) \right) \begin{prompt}= \answer{2}.\end{prompt} \]
\end{problem}



What if we get a function that isn't continuous?  
\begin{example}
	Evaluate: $\displaystyle \lim_{x\to 2} \dfrac{x^2-4}{x^2+x-6}$.
	\begin{explanation}
		Notice that when $x = 2$, the denominator $x^2+x-6$ becomes $0$, so $2$ is not in the domain of the function $\dfrac{x^2-4}{x^2+x-6}$, so we can't just plug into it.
		If we plug $x=2$ into the numerator, we also get $0$, so this is an indeterminate form of type \zeroOverZero.  In order to evaluate this limit, we need to do some algebra to simplify.
		Let's factor!
		\begin{align*}
			\dfrac{x^2-4}{x^2+x-6} &= \dfrac{(x+2)(x-2)}{(x+3)(x-2)}\\
				&= \dfrac{(x+2)\cancel{(x-2)}}{(x+3)\cancel{(x-2)}}\\
				&= \dfrac{x+2}{x+3}
		\end{align*}
		\[ \lim_{x\to 2} \dfrac{x-4}{x^2+x-6} \begin{prompt}= \answer{\frac{4}{5}} .\end{prompt}  \]	
	\end{explanation}
\end{example}

In this example, we were able to use our algebra skills to replace the function with a discontinuity at $x=2$ with a function that is continuous there.  Try one on your own.
\begin{problem}
	Evaluate the limit.
	\[ \lim_{x\to -1} \frac{x^2+5x+4}{2x^2+5x+3} \begin{prompt}= \answer{3}. \end{prompt} \]
\end{problem}

Algebra doesn't always help, though.  The function in the next example is not continuous at $x=3$, but we cannot replace it by a function that is continuous there.
\begin{example}
	Evaluate: $\displaystyle \lim_{x\to 3} \dfrac{2^x \ln(x-1)}{(x-3)^3\cos(x)}$.
	\begin{explanation}
		What happens if we try to plug in $x=3$?  The denominator becomes $0$, so that won't work.  What happens to the numerator?
		$2^3 \ln(3-1) = 8 \ln 2$ is not zero, so this is a limit of form \numOverZero.  The one-sided limits will be either $\infty$ or $-\infty$ (giving vertical asymptotes in the graph),
		and we need to check both.
		
		Let's start with the right-hand limit $x \to 3^+$:
		The numerator $2^x \ln(x-1)$ is continuous and positive near $3$.  In the denominator, $\cos(x)$ is continuous and near $\cos(3)$ (and negative) for $x$ near $3$.
		The $(x-3)^3$ factor will be $0$ at $x=3$, but for $x > 3$, $(x-3)$ is positive.  That means the fraction will be $\dfrac{\textrm{positive}}{\textrm{positive}\cdot\textrm{negative}}$, so
		\[ \lim_{x\to 3^+} \dfrac{ 2^x \ln(x-1)}{(x-3)^3 \cos(x)} = -\infty. \]
		
		On to the left-hand limit $x \to 3^-$:
		The numerator $2^x \ln(x-1)$ is continuous and positive near $3$.  In the denominator, $\cos(x)$ is continuous and near $\cos(3)$ (and negative) for $x$ near $3$.  
		The only possible change would come from the $(x-3)^3$ factor.  Since we are looking at $x < 3$, we know $(x-3)$ is negative, so $(x-3)^3$ is negative.  The fraction
		will be $\dfrac{\textrm{positive}}{\textrm{negative}\cdot\textrm{negative}}$.  Then \[ \lim_{x\to 3^-} \dfrac{2^x \ln(x-1)}{(x-3)^3 \cos(x)} = \infty. \]
		
		Putting these together we see that $\displaystyle \lim_{x\to 3}\dfrac{ 2^x \ln(x-1)}{(x-3)^3 \cos(x)}$ does not exist.
	\end{explanation}
\end{example}
For \zeroOverZero forms, algebra will usually help us evaluate the limit we are interested in.  For \numOverZero forms, like in this previous example, we're expecting one-sided limits
of $\pm \infty$.  We use the signs of the individual factors to see whether each one-sided limit is either $+\infty$ or $-\infty$, then compare them to determine if we can say anything 
about the limit itself.

\begin{problem}
	Evaluate the limit.
	\[ \lim_{x \to 1^-} \frac{(3+x)^2 \sin(x)}{x(1-x)^5 e^x } \begin{prompt} = \answer{+\infty} .\end{prompt} \]
\end{problem}	

In all of the limits we've been talking about so far, the input variable is tending to a specific value.  If instead we allow it to grow (in either direction) without bound, we are talking about
`Limits at Infinity' rather than just `Limits'.  These limits at infinity are the tools we used to detect horizontal asymptotes.
\begin{example}
	Evaluate: $\displaystyle \lim_{x\to \infty} \frac{4x^3-6x+2}{5x^3+7x^2-3x+1}$.
	\begin{explanation}
		The numerator and denominator each tend to $\infty$ as $x$ tends to $\infty$, so this limit has form \inftyOverInfty, another indeterminate form.  We need to do some simplification to evaluate it.  
		The function is rational, so remember our trick.  Start by finding the largest degree term in the denominator.  Here that's an $x^3$-term.  Divide the numerator and denominator
		by $x^3$.
		\begin{align*}
			\lim_{x\to\infty} \frac{4x^3-6x+2}{5x^3+7x^2-3x+1} &= \lim_{x\to\infty} \frac{4x^3-6x+2}{5x^3+7x^2-3x+1} \cdot \frac{1/x^3}{1/x^3}\\
				&= \lim_{x\to\infty} \frac{\frac{4x^3}{x^3}-\frac{6x}{x^3}+\frac{2}{x^3}}{\frac{5x^3}{x^3}+\frac{7x^2}{x^3}-\frac{3x}{x^3}+\frac{1}{x^3}}\\
				&= \lim_{x\to\infty} \frac{4-\frac{6}{x^2}+\frac{2}{x^3}}{5+\frac{7}{x}-\frac{3}{x^2}+\frac{1}{x^3}}\\
				&= \lim_{x\to\infty}\frac{4}{5}\\
				&= \answer{\frac{4}{5}}.
		\end{align*}
	\end{explanation}
\end{example}

With a little work, this trick also helps with functions that are not rational.
\begin{example}
	Evaluate: $\displaystyle \lim_{x\to -\infty} \frac{3x+1}{\sqrt{x^2+9}}$.
	\begin{explanation}
		In the square root, there is a polynomial $x^2+9$.  Since $x$ is tending toward $-\infty$, we can rewrite the radical in the denominator.  
		\begin{align*}
			\sqrt{x^2+9} &= \sqrt{x^2\left(1 + \frac{9}{x^2} \right)}\\
				&= \sqrt{x^2} \sqrt{1+\frac{9}{x^2}}\\
				&= |x| \sqrt{1+\frac{9}{x^2}}\\
				&= -x \sqrt{1+\frac{9}{x^2} }.
		\end{align*}
		(Why $-x$ instead of $x$?  When would we have gotten $x$ instead?)

		If the denominator were really $-x$, then we would divide the numerator and denominator by $x$.  Let's see what happens if we do that anyway.
		\begin{align*}
			\lim_{x\to -\infty} \frac{3x+1}{\sqrt{x^2+9}} &= \lim_{x\to -\infty} \frac{3x+1}{\sqrt{x^2+9}} \cdot \frac{1/x}{1/x}\\
				&= \lim_{x\to -\infty} \frac{3+\frac{1}{x}}{\frac{1}{x} \cdot \sqrt{x^2+9}} \\
				&= \lim_{x\to -\infty} \frac{3+\frac{1}{x}}{\frac{1}{x} \cdot -x \sqrt{1 + \frac{9}{x^2}}} \\
				&= \lim_{x\to -\infty} \frac{3+\frac{1}{x}}{ - \sqrt{1 + \frac{9}{x^2}}} \\
				&= \lim_{x\to -\infty} \frac{3}{-\sqrt{1}}\\
				&= \answer{-3}.
		\end{align*}
	\end{explanation}
\end{example}

Let's try one that looks a bit different.
\begin{example}
	Evaluate: $\displaystyle \lim_{x\to\infty} \left(x - \sqrt{x^2+1} \right)$.
	\begin{explanation}
		It is clear that $\lim_{x\to\infty}x = \infty$ and $\lim_{x\to\infty} \sqrt{x^2+1} = \infty$.  That means this limit has form $\inftyMinusInfty$, which is indeterminate.  To evaluate,
		we'll need to rewrite it a bit.  We'll start by multiplying and dividing by the conjugate.
		\begin{align*}
			\lim_{x\to\infty}\left(x - \sqrt{x^2+1}\right) &= \lim_{x\to\infty}\left(x - \sqrt{x^2+1}\right) \cdot \frac{x + \sqrt{x^2+1}}{x+\sqrt{x^2+1}}\\
				&= \lim_{x\to\infty} \frac{\left(x-\sqrt{x^2+1}\right)\left(x+\sqrt{x^2+1}\right)}{x+\sqrt{x^2+1}}\\
				&= \lim_{x\to\infty}\frac{x^2 - \left(x^2+1\right)}{x+\sqrt{x^2+1}}\\
				&= \lim_{x\to\infty}\frac{-1}{x + \sqrt{x^2+1}}.
		\end{align*}
		The numerator is a constant $-1$ and the denominator tends to $\infty$.  That means
		\[ \lim_{x\to\infty} \left( x - \sqrt{x^2+1} \right) = \answer{0}. \]
	\end{explanation}	
\end{example}

Try one on your own.
\begin{problem}
	Evaluate the limit.
	\[  \lim_{x\to -\infty} \frac{4x^3+x}{5x^3+2\sqrt{x^6 + 1}} \begin{prompt}= \answer{\frac{4}{3}} .\end{prompt}\]
\end{problem}



\end{document}
