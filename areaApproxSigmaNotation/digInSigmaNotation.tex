\documentclass{ximera}

%\usepackage{todonotes}

\newcommand{\todo}{}

\usepackage{tkz-euclide}
\tikzset{>=stealth} %% cool arrow head
\tikzset{shorten <>/.style={ shorten >=#1, shorten <=#1 } } %% allows shorter vectors

\usepackage{tkz-tab}  %% sign charts
\usetikzlibrary{decorations.pathreplacing} 

\usetikzlibrary{backgrounds} %% for boxes around graphs
\usetikzlibrary{shapes,positioning}  %% Clouds and stars
\usetikzlibrary{matrix} %% for matrix
\usepgfplotslibrary{polar} %% for polar plots
\usetkzobj{all}
\usepackage[makeroom]{cancel} %% for strike outs
%\usepackage{mathtools} %% for pretty underbrace % Breaks Ximera
\usepackage{multicol}

\usepackage{polynom}



\usepackage[many]{tcolorbox}  %% for titled boxes
\newtcolorbox{xbox}[1]{%
    tikznode boxed title,
    enhanced,
    arc=0mm,
    interior style={white},
    attach boxed title to top center= {yshift=-\tcboxedtitleheight/2},
    fonttitle=\bfseries,
    colbacktitle=white,coltitle=black,
    boxed title style={size=normal,colframe=white,boxrule=0pt},
    title={#1}}


\usepackage{array}
\setlength{\extrarowheight}{+.1cm}   
\newdimen\digitwidth
\settowidth\digitwidth{9}
\def\divrule#1#2{
\noalign{\moveright#1\digitwidth
\vbox{\hrule width#2\digitwidth}}}





\newcommand{\RR}{\mathbb R}
\newcommand{\R}{\mathbb R}
\newcommand{\N}{\mathbb N}
\newcommand{\Z}{\mathbb Z}

%\renewcommand{\d}{\,d\!}
\renewcommand{\d}{\mathop{}\!d}
\newcommand{\dd}[2][]{\frac{\d #1}{\d #2}}
\newcommand{\pp}[2][]{\frac{\partial #1}{\partial #2}}
\renewcommand{\l}{\ell}
\newcommand{\ddx}{\frac{d}{\d x}}
\newcommand{\ddt}{\frac{d}{\d t}}

\newcommand{\zeroOverZero}{\ensuremath{\boldsymbol{\tfrac{0}{0}}}}
\newcommand{\inftyOverInfty}{\ensuremath{\boldsymbol{\tfrac{\infty}{\infty}}}}
\newcommand{\zeroOverInfty}{\ensuremath{\boldsymbol{\tfrac{0}{\infty}}}}
\newcommand{\zeroTimesInfty}{\ensuremath{\small\boldsymbol{0\cdot \infty}}}
\newcommand{\inftyMinusInfty}{\ensuremath{\small\boldsymbol{\infty - \infty}}}
\newcommand{\oneToInfty}{\ensuremath{\boldsymbol{1^\infty}}}
\newcommand{\zeroToZero}{\ensuremath{\boldsymbol{0^0}}}
\newcommand{\inftyToZero}{\ensuremath{\boldsymbol{\infty^0}}}



\newcommand{\numOverZero}{\ensuremath{\boldsymbol{\tfrac{\#}{0}}}}
\newcommand{\dfn}{\textbf}
%\newcommand{\unit}{\,\mathrm}
\newcommand{\unit}{\mathop{}\!\mathrm}
\newcommand{\eval}[1]{\bigg[ #1 \bigg]}
\newcommand{\seq}[1]{\left( #1 \right)}
\renewcommand{\epsilon}{\varepsilon}
\renewcommand{\iff}{\Leftrightarrow}

\DeclareMathOperator{\arccot}{arccot}
\DeclareMathOperator{\arcsec}{arcsec}
\DeclareMathOperator{\arccsc}{arccsc}
\DeclareMathOperator{\si}{Si}
\DeclareMathOperator{\proj}{proj}
\DeclareMathOperator{\scal}{scal}


\newcommand{\tightoverset}[2]{% for arrow vec
  \mathop{#2}\limits^{\vbox to -.5ex{\kern-0.75ex\hbox{$#1$}\vss}}}
\newcommand{\arrowvec}[1]{\tightoverset{\scriptstyle\rightharpoonup}{#1}}
\renewcommand{\vec}{\mathbf}
\newcommand{\veci}{\vec{i}}
\newcommand{\vecj}{\vec{j}}
\newcommand{\veck}{\vec{k}}
\newcommand{\vecl}{\boldsymbol{\l}}

\newcommand{\dotp}{\bullet}
\newcommand{\cross}{\boldsymbol\times}
\newcommand{\grad}{\boldsymbol\nabla}
\newcommand{\divergence}{\grad\dotp}
\newcommand{\curl}{\grad\cross}
%\DeclareMathOperator{\divergence}{divergence}
%\DeclareMathOperator{\curl}[1]{\grad\cross #1}


\colorlet{textColor}{black} 
\colorlet{background}{white}
\colorlet{penColor}{blue!50!black} % Color of a curve in a plot
\colorlet{penColor2}{red!50!black}% Color of a curve in a plot
\colorlet{penColor3}{red!50!blue} % Color of a curve in a plot
\colorlet{penColor4}{green!50!black} % Color of a curve in a plot
\colorlet{penColor5}{orange!80!black} % Color of a curve in a plot
\colorlet{fill1}{penColor!20} % Color of fill in a plot
\colorlet{fill2}{penColor2!20} % Color of fill in a plot
\colorlet{fillp}{fill1} % Color of positive area
\colorlet{filln}{penColor2!20} % Color of negative area
\colorlet{fill3}{penColor3!20} % Fill
\colorlet{fill4}{penColor4!20} % Fill
\colorlet{fill5}{penColor5!20} % Fill
\colorlet{gridColor}{gray!50} % Color of grid in a plot

\newcommand{\surfaceColor}{violet}
\newcommand{\surfaceColorTwo}{redyellow}
\newcommand{\sliceColor}{greenyellow}




\pgfmathdeclarefunction{gauss}{2}{% gives gaussian
  \pgfmathparse{1/(#2*sqrt(2*pi))*exp(-((x-#1)^2)/(2*#2^2))}%
}


%%%%%%%%%%%%%
%% Vectors
%%%%%%%%%%%%%

%% Simple horiz vectors
\renewcommand{\vector}[1]{\left\langle #1\right\rangle}


%% %% Complex Horiz Vectors with angle brackets
%% \makeatletter
%% \renewcommand{\vector}[2][ , ]{\left\langle%
%%   \def\nextitem{\def\nextitem{#1}}%
%%   \@for \el:=#2\do{\nextitem\el}\right\rangle%
%% }
%% \makeatother

%% %% Vertical Vectors
%% \def\vector#1{\begin{bmatrix}\vecListA#1,,\end{bmatrix}}
%% \def\vecListA#1,{\if,#1,\else #1\cr \expandafter \vecListA \fi}

%%%%%%%%%%%%%
%% End of vectors
%%%%%%%%%%%%%

%\newcommand{\fullwidth}{}
%\newcommand{\normalwidth}{}



%% makes a snazzy t-chart for evaluating functions
%\newenvironment{tchart}{\rowcolors{2}{}{background!90!textColor}\array}{\endarray}

%%This is to help with formatting on future title pages.
\newenvironment{sectionOutcomes}{}{} 



%% Flowchart stuff
%\tikzstyle{startstop} = [rectangle, rounded corners, minimum width=3cm, minimum height=1cm,text centered, draw=black]
%\tikzstyle{question} = [rectangle, minimum width=3cm, minimum height=1cm, text centered, draw=black]
%\tikzstyle{decision} = [trapezium, trapezium left angle=70, trapezium right angle=110, minimum width=3cm, minimum height=1cm, text centered, draw=black]
%\tikzstyle{question} = [rectangle, rounded corners, minimum width=3cm, minimum height=1cm,text centered, draw=black]
%\tikzstyle{process} = [rectangle, minimum width=3cm, minimum height=1cm, text centered, draw=black]
%\tikzstyle{decision} = [trapezium, trapezium left angle=70, trapezium right angle=110, minimum width=3cm, minimum height=1cm, text centered, draw=black]


\outcome{Write a sum into sigma notation.}
\outcome{Evaluate sums written in sigma notation.}
\outcome{Recognize sigma notation.}


\title[Dig-In:]{Sigma Notation}


\begin{document}
\begin{abstract}

\end{abstract}
\maketitle

We've seen that our area approximations get better and better when we take more 
and more rectangles.  When the number of rectangles we have to work with is small 
(under 10 or so), we can work everything out by hand.  However, if we need to use
more, it becomes cumbersome. 

\section{Sigma Notation}
\emph{Sigma notation} is a way of writing a sum of many terms, in a concise form.  A sum in sigma notation looks something like this:

\[ \sum_{k=1}^{5} 3k \]

The $\Sigma$ (sigma) indicates that a sum is being taken.  The variable $k$ is called the \emph{index} of the sum.  The numbers at the top and bottom of the $\Sigma$
are called the \emph{upper and lower limits} of the summation.  In this case, the upper limit is $5$, and the lower limit is $1$.  The notation means that we will take
every integer value of $k$ between $1$ and $5$ (so $1$, $2$, $3$, $4$, and $5$) and plug them each into the summand formula (here that formula is $3k$).
Then those are all added together.
\[ \sum_{k=1}^{5} 3k= 3\cdot 1 + 3\cdot 2 + 3\cdot 3 + 3\cdot 4 + 3\cdot 5 = 45 \]



\begin{example}
	Write out what is meant by the following:
	\[ \sum_{k=0}^{3} \frac{1}{k+1}\]
	\begin{explanation}
		Here, the index $k$ takes the values $0$, $1$, $2$, and $3$.  We'll plug those each into $\frac{1}{k+1}$ and add them together.
		\[\sum_{k=0}^3 \frac{1}{k+1} = \frac{1}{\answer{0}+1} + \frac{1}{\answer{1}+1} + \frac{1}{\answer{2}+1} + \frac{1}{\answer{3}+1} \]
	\end{explanation}
\end{example}


 

\begin{example}
	Write out what is meant by the following:
	\[ \sum_{i=1}^{8} \left(-1\right)^i \]
	\begin{explanation}
		The index variable here is written as $i$ instead of $k$.  That's ok.  The most common variables to use for indexes include $i$, $j$, $k$, $m$, and $n$.
		\begin{align*}
			\sum_{i=1}^8 (-1)^i &= (-1)^{\answer{1}} + (-1)^{\answer{2}} + (-1)^{\answer{3}} + (-1)^{\answer{4}} + (-1)^{\answer{5} }+ (-1)^{\answer{6}} + (-1)^{\answer{7}} + (-1)^{\answer{8}} \\
				&= -1 + 1 + -1 + 1 + -1 + 1 + -1 + 1\\
				&= \answer{0}
		\end{align*}	
	\end{explanation}
\end{example}
 

Try one on your own.
\begin{question}
	Write out what is meant by the following (no need to simplify):
	\[ \sum_{n=-1}^{4} \sqrt{n+1} =\answer{\sqrt{0}+\sqrt{1}+\sqrt{2}+\sqrt{3}+\sqrt{4}+\sqrt{5}}\]
\end{question}
 

Let's try going the other way around.
\begin{example}
	Write the following sum in sigma notation.
	\[ 2 + 4 + 6 + 8 + \ldots + 22 + 24 \]
	\begin{explanation}
		Notice that we can factor a $2$ out of each term to rewrite this sum as
		\[ \answer{2}\cdot 1 + \answer{2} \cdot 2 + \answer{2}\cdot 3 + \answer{2} \cdot 4 + \ldots + \answer{2}\cdot 11 + \answer{2}\cdot 12 \]
		That means that we are adding together $\answer{2}$ times every number between $1$ and $12$.  If we use $k$ as our index, the sigma notation could be
		\[ \sum_{k=1}^{12} \answer{2k} \]
		There is no need to use $k$ as our index variable.  We could have just as easily used $m$ or $j$ instead.
		\[ \sum_{k=1}^{12} 2k = \sum_{m=1}^{12} 2m = \sum_{j=1}^{12} 2j\]  
		Notice, that these are NOT the same as
		\[ \sum_{k=1}^{12} 2m \]
	\end{explanation}
\end{example}
 

\begin{example}
	Write the following sum in sigma notation.
	\[ 1 - \frac{1}{2} + \frac{1}{4} - \frac{1}{8} + \ldots - \frac{1}{64} + \frac{1}{128}  \]
	\begin{explanation}
		This one is a little more complicated.  We'll worry about the signs later, first we'll deal with the numbers themselves.  Do you notice a pattern in the terms?
		Sure, we get from one term to the next term by dividing by 2.  That is:
		\begin{align*}
			1 &= \frac{1}{2^0} = \left(\frac{1}{2}\right)^0\\
			\frac{1}{2} &= \frac{1}{2^1} = \left(\frac{1}{2}\right)^{1}\\
			\frac{1}{4} &= \frac{1}{2^2} = \left(\frac{1}{2}\right)^{2}\\
				 ...  \\
			 \frac{1}{128} &= \frac{1}{2^7} = \left(\frac{1}{2}\right)^{7}
		\end{align*}
		If we call our index variable $k$, then $k$ should go from $0$ to $\answer{7}$, and the numbers themselves are just $\left(\frac{1}{2}\right)^{\answer{k}}$.  Now we need to deal with the signs.
		We say above that $(-1)^k$ will alternate between $+1$ and $-1$.  That means, if we multiply the terms we just found by $(-1)^k$, they will alternate between $+$ and $-$.  
		We are starting with $k=0$, so $(-1)^0 = +1$ will give us the alternation starting at the sign we want.
		\[ \sum_{k=0}^{7} (-1)^k \left(\frac{1}{2}\right)^{k} = \sum_{k=0}^7 \left( \answer{\frac{-1}{2}} \right)^k \]
	\end{explanation}
\end{example}
  
 
Try one on your own.
\begin{example}
 	Write the following sum in sigma notation.
 	\[ \frac{1}{\sqrt{2}} - \frac{2}{\sqrt{3}} + \frac{3}{\sqrt{4}} - \frac{4}{\sqrt{5}}+ \ldots + \frac{51}{\sqrt{52}} - \frac{52}{\sqrt{53}} \]
	\[ \sum_{k=2}^{53} \left(  \answer{-\frac{k-1}{\sqrt{k}} } \right) \]

\end{example}
   

 
 \section{Calculating with sigma notation}
 We want to use sigma notation to simplify our calculations.  To do that, we will need to know some basic sums.  
 First, let's talk about the sum of a constant.  (Notice here, that our upper limit of summation is $n$.  $n$ is not the index 
 variable, here, but the highest value that the index variable will take.)
 \[ \sum_{k=1}^n C \] 
 This is a sum of $n$ terms, each of them having a value $C$.  That is, we are adding $n$ copies of $C$.  This sum is just $nC$.
 The other basic sums that we need are much more complicated to derive.  Rather than explaining where they come from, we'll just give you
 a list of the final formulas, that you can use.
 
 \begin{align*}
	\sum_{k=1}^n C &= nC \\ \\
	\sum_{k=1}^n k &= \frac{n(n+1)}{2} \\ \\
	\sum_{k=1}^n k^2 &= \frac{n(n+1)(2n+1)}{6} \\ \\
	\sum_{k=1}^n k^3 &= \left(\frac{n(n+1)}{2}\right)^2
\end{align*} 


Now that we have this list, let's use them to compute.
\begin{example}
	Find the value of the sum $\displaystyle \sum_{k=1}^{5} 9$.
	\begin{explanation}
		This is just the sum of a constant, with $C=\answer{9}$ and $n=\answer{5}$.  The value is $nC=\answer{45}$.	
	\end{explanation}
\end{example}

 

\begin{example}
	Find the value of the sum $\displaystyle \sum_{k=1}^{100} k$.
	\begin{explanation}
		This is the sum $1+2+3+ \ldots + 100$.  According to Formula 2 above (with $n=\answer{100}$), this is $\frac{\answer{100}(\answer{101})}{2} = \answer{5050}$.	
	\end{explanation}
\end{example}

 

 Because sigma notation is just a new way of writing addition, the usual properties of addition still apply, but a couple of the important ones look a little different.
\[ \text{Commutativity:} \quad \sum_{k=1}^{n} (a_k + b_k) = \sum_{k=1}^n a_k + \sum_{k=1}^n b_k \]
\[ \text{Distribution:}  \quad \sum_{k=1}^{n} c \cdot a_k = c \sum_{k=1}^n a_k \]


\begin{example}
	Find the value of the sum $\displaystyle \sum_{k=1}^{10} \left(2k^2+5\right)$.
	\begin{explanation}
		First, we'll use the properties above to split this into two sums, then factor the $2$ out of the first sum.
		\begin{eqnarray*}
			\sum_{k=1}^{10} \left(2k^2+5\right) &=& \sum_{k=1}^{10} 2k^2 + \sum_{k=1}^{10} 5\\
				&=& 2 \sum_{k=1}^{10} \answer{k^2} + \sum_{k=1}^{10} 5
		\end{eqnarray*}
		The two sums we have left, can be found using formulas 1 and 3 above!
		
		 We see that $\displaystyle \sum_{k=1}^{10} k^2 = \frac{\answer{10}(\answer{10}+1)(2\cdot \answer{10}+1)}{6} = \answer{385}$.  Similarly, $\displaystyle \sum_{k=1}^{10} 5 = \answer{50}$.
		
		 Putting all that together, $\displaystyle \sum_{k=1}^{10}\left(2k^2+5\right) = 2 \cdot 385 + 50 = \answer{820}$.
	\end{explanation}
\end{example}
 

\begin{example}
	Find the value of the sum $\displaystyle \sum_{k=1}^{200} \left(2k^3-6k^2+3\right)$.
	\begin{explanation}
		Let's use the same approach as in the previous example.  First, we'll use the properties to split this into individual sums, then factor out the coefficients.  
		After that, we'll use the formulas above to evaluate it.
		\begin{eqnarray*}
			\sum_{k=1}^{200} \left(2k^3-6k^2+3\right) &=& \sum_{k=1}^{200} 2k^3 - \sum_{k=1}^{200} 6k^2 + \sum_{k=1}^{200}3\\
				&=& 2 \sum_{k=1}^{200} \answer{k^3} - 6 \sum_{k=1}^{200} \answer{k^2} + \sum_{k=1}^{200}\answer{3}\\
				&=& 2 \left( \frac{200(201)}{2}\right)^2 - 6 \left( \frac{200(201)(401)}{6}\right) + 200 \cdot 3\\
%				&=& 2 \cdot 404010000 - 6 \cdot 2686700 + 200 \cdot 3\\
				&=& \answer{791900400}
		\end{eqnarray*}
	The numbers in this example were horribly ugly, but we were able to evaluate the sum without having to actually calculate all 200 terms, then add them all up.
	In 4 small lines, we were able to add 200 numbers.
	
	\end{explanation}
\end{example}

 

Try one on your own.
\begin{problem}
	Find the value of the sum $\displaystyle \sum_{k=1}^{50} \left( 4k^2-18k + 2(-1)^k \right)$
	
	\[ \sum_{k=1}^{50} \left( 4k^2 - 18k + 2(-1)^k \right) = \answer{148750} \]
\end{problem}
  
\end{document}
